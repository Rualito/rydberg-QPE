\documentclass[margin=0.2in]{standalone}
\usepackage{physics} 

\usepackage{pgfplots}
\usepackage{tikz-3dplot}
\usetikzlibrary{angles, quotes}

\tdplotsetmaincoords{70}{60}
\pgfplotsset{compat=newest}
 
\begin{document}
 
\begin{tikzpicture}[tdplot_main_coords, scale = 5.5]
    % Draw shaded circle
    % \shade[ball color = lightgray,
    % opacity = 0.5
    % ] (0,0,0) circle (1cm);
    
    \draw (0, 0) node[circle, fill, inner sep=1] (orig) {};
    % draw arcs 

    % \tdplotsetrotatedcoords{0}{0}{0};
    % \draw[dashed,
    %     tdplot_rotated_coords,
    %     gray
    % ] (0,0,0) circle (1);
    
    % \tdplotsetrotatedcoords{90}{90}{90};
    % \draw[dashed,
    %     tdplot_rotated_coords,
    %     gray
    % ] (1,0,0) arc (0:360:1);
    
    % \tdplotsetrotatedcoords{0}{90}{90};
    % \draw[dashed,
    %     tdplot_rotated_coords,
    %     gray
    % ] (1,0,0) arc (0:360:1);

    % \tdplotsetrotatedcoords{-45}{-90}{-90};
    % \draw[dashed,
    %     tdplot_rotated_coords,
    %     gray
    % ] (1,0,0) arc (0:360:1);

    % \tdplotsetrotatedcoords{0}{0}{0};
    % \draw[dashed,
    %     tdplot_rotated_coords,
    %     gray
    % ] (1,0,0) arc (0:45:1);
    % Create a point (P)
    \coordinate (P) at ({1/sqrt(3)},{1/sqrt(3)},{1/sqrt(3)});

    \coordinate (A) at (0,0,0);
    \coordinate (B) at (1,0,0);
    \coordinate (C) at ({2/3},{2/3},0);
    \coordinate (D) at ({1/2},{1/2},{1/2});

    \draw  node[circle, fill, inner sep=1, label=left:$x_1$] at (A)  {};
    \draw  node[circle, fill, inner sep=1, label=below:$x_2$] at (B)  {};
    \draw  node[circle, fill, inner sep=1, label=right:$x_3$] at (C)  {};
    \draw  node[circle, fill, inner sep=1, label=above:$x_4$] at (D)  {};

    \draw[thick, -stealth] (A) -- (D) node[above, midway] {$R^3$};
    \draw[thick, -stealth] (A) -- (B) node[below, midway] {$R$};
    \draw[thick, -stealth] (A) -- (C) node[above, midway] {$R^2$};
    \draw[thick, -stealth] (B) -- (C) node[below right, midway] {$R$};
    \draw[thick, -stealth] (B) -- (D) node[right, midway] {$R^2$};
    \draw[thick, -stealth] (C) -- (D) node[above, midway] {$R$};

    % \draw ({1/sqrt(3)},{1/sqrt(3)},{1/sqrt(3)}) node[right] (P1) {$\ket{\psi}$};

    % \draw (0,0,1) node[circle, fill, inner sep=1, label=right:$\ket{0}$] (k0) {};
    % \draw (0,0,-1) node[circle, fill, inner sep=1, label=right:$\ket{1}$] (k1) {};
    % \draw[gray, thin, dashed] (orig) -- (k1) ;
    
    % \draw node[circle, fill, inner sep=1] at (P) (P_t) {$P$};
    
    % Projection of the point on X and y axes
    % \draw[thin, dashed] (P) -- ({1/sqrt(3)},{1/sqrt(3)},0);
    % \draw[thin, dashed]  (0,0,0) -- ({1/sqrt(2)},{1/sqrt(2)},0) node (rho) {};
    
    % \draw[thin, dashed] ({1/sqrt(3)},{1/sqrt(3)},0) --++
    % (0,{-1/sqrt(3)},0);
    % \draw[thin, dashed] ({1/sqrt(3)},{1/sqrt(3)},0) --++
    % ({-1/sqrt(3)},0,0);
    
    % Axes in 3 d coordinate system
    % \draw[-stealth] (0,0,0) -- (1.0,0,0) node[below] (x) {$x$};
    % \draw[-stealth] (0,0,0) -- (0,1.0,0) node[right] (y) {$y$};
    % \draw[-stealth] (0,0,0) -- (0,0,1.0) node[above] (z) {$z$};
    % \draw[dashed, gray] (0,0,0) -- (-1,0,0);
    % \draw[dashed, gray] (0,0,0) -- (0,-1,0);
    
    % Line from the origin to (P)
    

    % \pic [draw=black, text=black, ->, "$\phi$", angle radius=20] {angle = x--orig--rho};
    % \pic [draw=black, text=black, <-, "$\theta$", angle eccentricity=1.4] {angle = P--orig--z};
  

\end{tikzpicture}
 
\end{document}